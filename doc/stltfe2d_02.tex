\documentclass[12pt, letterpaper]{article}
\usepackage{graphicx} % Required for inserting images
\usepackage{amsmath}
\usepackage{mdframed}
\usepackage{amssymb}

\title{telegraph equations solved with MFEM library using MFEM}
\author{Denis Lachapelle}
\date{December 2025}

\setlength{\parindent}{0pt}

\begin{document}
	
\maketitle


\tableofcontents


\section{Introduction}
This document explains the telegrapher equations simulation using finite element method based on MFEM library.

It is done in 2D: The transmission line is along x-axis, while time is along y-axis.

It is not working yet.

\section{Theory}
I think we can solve the equations assuming time t is another space dimension, so the first order coupled telegrapher equations will become like 2D PDE. There won't be time stepping since the space t will be all solved at once.\\

\subsection{From Telegrah PDE to Matrix form}

The telegraph equations are:

\begin{equation}\frac{\partial{V}}{\partial{x}} + L \frac{\partial{I}}{\partial{t}} + R I = 0\end{equation}


\begin{equation}\frac{\partial{I}}{\partial{x}} + C \frac{\partial{V}}{\partial{t}} + G V = 0\end{equation}

Assume the current $I$ is going from left to right and that the source is connected to the left of the transmission line and the load to the right.\\

Assume left is position 0.\\

Now we change t for y, a space dimensions.\\

Going to weak form using space H1 for V and  H1 for I . 

\begin{equation}\int_\Omega(\frac{\partial{V(x, y)}}{\partial{x}} + L \frac{\partial{I(x,y)}}{\partial{y}} + R I(x,y)) \phi_i d\Omega = 0\end{equation}

\begin{equation}\int_\Omega(\frac{\partial{I(x,y)}}{\partial{x}} + C \frac{\partial{V(x,y)}}{\partial{y}} + G V(x,y)) \psi_i d\Omega = 0\end{equation}

and then


\begin{equation}
	\int_\Omega\frac{\partial{V(x,y)}}{\partial{x}} \phi_i
	+ \int_\Omega L \frac{\partial{I(x,y)}}{\partial{y}} \phi_i
	+ \int_\Omega R I(x,y) \phi_i
	= 0
\end{equation}

From the above equation we can select the proper bilinear integrator for each terms.

\begin{itemize}
	\item Bilinear DerivativeIntegrator in x direction.
	\item MixedBilinear MixedDirectionalDerivativeIntegrator in y direction.
	\item MixedScalarMassIntegrator.
\end{itemize}


\begin{equation}
	\int_\Omega\frac{\partial{I(x,y)}}{\partial{x}} \psi_i
	+ \int_\Omega C \frac{\partial{V(x,y)}}{\partial{y}} \psi_i
	+  \int_\Omega G V(x,y) \psi_i
	= 0
\end{equation}

From the above equation we can select the proper bilinear integrator for each terms.

\begin{itemize}
	\item Bilinear DerivativeIntegrator in x direction.
	\item MixedBilinear MixedDirectionalDerivativeIntegrator in y direction.
	\item MixedScalarMassIntegrator.
\end{itemize}

With the following definitions:
\begin{itemize}
	\item $D_x$: DerivativeIntegrator in x direction
	\item $D_y$: MixedDirectionalDerivativeIntegrator in y direction.
	\item $M$: MixedScalarMassIntegrator.
\end{itemize}

We can form the following block matrix.

\begin{equation}
	\begin{bmatrix}
		D_x & L D_y + R M  \\
		C D_y + G M   & D_x
	\end{bmatrix}
	\begin{bmatrix}
		V\\
		I
	\end{bmatrix}
	=
	\begin{bmatrix}
		bV\\
		bI
	\end{bmatrix}	
\end{equation}

Note that for now bV and bI are zero.

For future reference define A11 to A22:

\begin{equation}
	\begin{bmatrix}
		A11 & A12  \\
		A21  & A22
	\end{bmatrix}	
\end{equation}


\subsection{Boundary Condition applying Vs Rs}

Boundary conditions shall be applied I(x=0, y) and V(x=0, y) for $0 < y < T$, to apply the Vs Rs source.\\

$ I(0,y) = \frac{V_s(y) - V(0,y)}{Rs} $ \\

This equation (in weak form) can be added to the block matrix to enforce the VSRS boundary condition.\\


$\int_{\Gamma_s} Rs I(0,y) \phi d\Gamma + \int_{\Gamma_s } V(0,y) \phi d\Gamma = \int_{\Gamma_s } V_s(y) \phi d\Gamma$ \\

Where $\Gamma_s$ is the left boundary, $x=0, 0 <y < T$.

The source shall be view as a current that is pushed to the transmission line.\\

With the following definitions:
\begin{itemize}
	\item $B_1$: BoundaryMassIntegrator with coeff 1.0.
	\item $B_2$: BoundaryMassIntegrator with coeff Rs.
	\item $L_1$: BoundaryLFIntegrator with FunctionCoeff  Vs(y).
\end{itemize}

Modify the block matrix as follow:

On first row add B1 to A11, add B2 to A12 and add $L_1$ to rhs bV.


\subsection{Boundary Condition applying $R_L$}

Boundary conditions shall be applied I(x=Len, y) and V(x=Len, y) for $0 < y < T$, to apply the load $R_L$.\\

$ I(Len,y) = \frac{V(Len,y)}{R_L} $ \\

This equation (in weak form) can be added to the block matrix to enforce $R_L$ boundary condition.\\


$\int_{\Gamma_u} R_L I(Len,y) \phi d\Gamma - \int_{\Gamma_u } V(Len,y) \phi d\Gamma = 0$ \\

Where $\Gamma_u$ is the right boundary, $x=Len, 0 <y < T$.

The load shall be view as a current sink.\\

With the following definitions:
\begin{itemize}
	\item $B_3$: BoundaryMassIntegrator with coeff -1.0.
	\item $B_4$: BoundaryMassIntegrator with coeff $R_L$.
\end{itemize}

Modify the block matrix as follow:

On first row add B3 to A11, add B4 to A12.


\subsection{Boundary Conditions on x-axis at t=0}

Since the transmission line is at rest $V(x,y=0) = 0$ and $I(x,y=0) = 0$ for $0 < x < Len$.


\section{MFEM Inplementation}

Jan 8, 2026 I decide to abandon this approach I do not succeed in making the system converging after many trials. I also try using Lagrange Multiplier without success. I will go back to solving the telegrapher equations in time.

	
	
\end{document}
