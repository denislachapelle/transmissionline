\documentclass[12pt, letterpaper]{article}
\usepackage{graphicx} % Required for inserting images
\usepackage{float} % For [H]
\usepackage{amsmath}
\usepackage{mdframed}

\title{Telegrapher's Eequations Solved with MFEM Library}
\author{Denis Lachapelle}
\date{February 2025}

\setlength{\parindent}{0pt}

\begin{document}

\maketitle

\section{Introduction}
This document explains the telegrapher's equations simulation using finite differences method; one first version using python (numpy and scipy) and a second version based on MFEM library in C++.


\section{Theory, Finite Differences}
\subsection{Continuous Time Equations}

The telegraph equations are:

\begin{equation}\frac{\partial{V}}{\partial{x}} + L \frac{\partial{I}}{\partial{t}} + R I = 0\end{equation}


\begin{equation}\frac{\partial{I}}{\partial{x}} + C \frac{\partial{V}}{\partial{t}} + G V = 0\end{equation}

\subsection{Discretization}

\begin{enumerate}
\item Assume the transmission line is divided in N segments (nbrSeg) with nodes 0 to N. There is N+1 nodes. So nodes are located at $n \Delta x$.
\item Time is sampled at $k \Delta T$.
\item Assume the $\frac{\partial V(t, x)}{\partial x}$ and $\frac{\partial I(t, x)}{\partial x}$ are constant over each segment.
\end{enumerate}

So at a given time $k \Delta T$ we can express the $\frac{\partial V(t, x)}{\partial x}$ and $\frac{\partial I(t, x)}{\partial x}$ as difference  $\frac{V(k \Delta T, (n+1) \Delta x) - V(k \Delta T, (n) \Delta x)}{\Delta x}$ and  $\frac{I(k \Delta T, (n) \Delta x) - I(k \Delta T, (n-1) \Delta x)}{\Delta x}$.\\

Rewrite the equations 1 and 2 with the $\frac{\partial}{\partial t}$ on the left side.

\begin{equation} \frac{\partial{I}}{\partial{t}} = - \frac{R I}{L} - \frac{1}{L} \frac{\partial{V}}{\partial{x}} \end{equation}


\begin{equation} \frac{\partial{V}}{\partial{t}} = -\frac{G V}{C} - \frac{1}{C} \frac{\partial{I}}{\partial{x}}\end{equation}

We can write matrix equations....

\begin{equation}
\frac{\partial{I}}{\partial{t}} 
=
	\begin{bmatrix}
		Dv & Ri
	\end{bmatrix}
	\begin{bmatrix}
		V^k \\
		I^k \\
	\end{bmatrix}
\end{equation}

\begin{equation}
	\frac{\partial{V}}{\partial{t}} 
	=
	\begin{bmatrix}
		Gv & Di
	\end{bmatrix}
	\begin{bmatrix}
		V^k \\
		I^k \\
	\end{bmatrix}
\end{equation}

The two equations above can be written as a single matrix equation...

\begin{equation}
    \begin{bmatrix}
    	\frac{\partial{V}}{\partial{t}} \\
    	\frac{\partial{I}}{\partial{t}} 
    \end{bmatrix}	
	=
	\begin{bmatrix}
		Gv Di \\
		Dv Ri
	\end{bmatrix}
	\begin{bmatrix}
		V^k \\
		I^k \\
	\end{bmatrix}
\end{equation}


The Dv matrix ix nbrSeg x (nbrSeg+1) as example for nbrSeg=5 ...

\begin{equation}
	\frac{-1}{L h}
	\begin{bmatrix}
	   -1 &  1 & 0 & 0 & 0 &0 \\
	    0 & -1 & 1 & 0 & 0 &0 \\
	    0 &  0 & -1 & 1 & 0 &0 \\
	    0 &  0 & 0 & -1 & 1 & 0 \\
	    0 &  0 & 0& 0 & -1 & 1  \\
	   
	\end{bmatrix}
\end{equation}

Notice the matrix is complete in the sense that the derivative is fully estimated for each row, this correspond to the fact that each current segment is enclosed by two voltage nodes.\\

The Di matrix ix (nbrSeg+1) x nbrSeg as example for nbrSeg=5 ...

\begin{equation}
	\frac{-1}{C h}
	\begin{bmatrix}
		 1 &  0 &  0 &  0 &  0  \\
		-1 &  1 &  0 &  0 &  0  \\
		 0 & -1 &  1 &  1 &  0  \\
		 0 &  0 & -1 &  1 &  1  \\
		 0 &  0 &  0 & -1 &  1  \\
		 0 &  0 &  0 &  0 & -1   \\
	\end{bmatrix}
\end{equation}

Notice the matrix is not complete in the sense that the derivative are not fully estimated for first and last row, this correspond to the fact that for the end voltage nodes there is no source and load current. That will have to be considered in the boundary conditions.\\

Gv is (nbrSeg+1) x (nbrSeg+1) identity matrix scale by either -G/C.\\

Ri is nbrSeg x nbrSeg identity matrix scale by either -R/L.\\

\subsection{Adding a Source}

To add the source with a series resistor we should write the equation for $\frac{\partial{V_0}}{\partial{t}}$ which one will replace the first row equations. The equation is the same as equation 4 with added current from Rs.

\begin{equation} \frac{\partial{V_0}}{\partial{t}} = -\frac{G V_0}{C} - \frac{1}{C} \frac{\partial{I}}{\partial{x}} + \frac{Vs}{Rs C h} - \frac{V_0}{Rs C h}
\end{equation}



This done by adding a voltage node Vs, a row to compute Vs.

\begin{itemize}
\item the first row is all zero since $\frac{\partial V_s}{\partial t} = 0$ and in fact is not used.
\item the second row is 1/(Rs C h) -1/(Rs C h) 0 0 ... 0.
\item the last sub matrix is all 0.0.
\end{itemize}

The resulting matrix equation is ...


\begin{equation}
\begin{bmatrix}
	Vs \\
	\frac{\partial{V}}{\partial{t}} \\
	\frac{\partial{I}}{\partial{t}} 
\end{bmatrix}	
=
\left[ 
\begin{bmatrix}
	0 & 0 & 0 \\
	0 & Gv & Di \\
	0 & Dv & Ri \\
\end{bmatrix}
+
\begin{bmatrix}
	1 & 0 & 0 \\
	\frac{1}{Rs C} & \frac{-1}{Rs C} & 0 \\
	0 & 0 & 0 \\
\end{bmatrix}
\right]
\begin{bmatrix}
	Vs \\
	V^k \\
	I^k \\
\end{bmatrix}
\end{equation}

\subsection{Adding a Load}

Now we need to add the right boundary which is a load Rl. The last voltage node will get influenced by Rl, the equation becomes ...

\begin{equation} \frac{\partial{V_N}}{\partial{t}} = -\frac{G V_N}{C} - \frac{1}{C} \frac{\partial{I_{N-1}}}{\partial{x}} - \frac{V_N}{Rl C h}
\end{equation}

The resulting matrix equation is ...


\begin{equation}
\begin{bmatrix}
	Vs \\
	\frac{\partial{V}}{\partial{t}} \\
	\frac{\partial{I}}{\partial{t}} 
\end{bmatrix}	
=
\left[ 
\begin{bmatrix}
	0 & 0 & 0 \\
	0 & Gv & Di \\
	0 & Dv & Ri \\
\end{bmatrix}
+
\begin{bmatrix}
	1 & 0 & 0 \\
	\frac{1}{Rs C h} & \frac{-1}{Rs C h} & 0 \\
	0 & 0 & 0 \\
\end{bmatrix}
+
\begin{bmatrix}
	0 & 0 & 0 \\
	0 & \frac{-1}{Rs C h} & 0 \\
	0 & 0 & 0 \\
\end{bmatrix}
\right]
\begin{bmatrix}
	Vs \\
	Vk \\
	Ik 
\end{bmatrix}
\end{equation}

The non zero value is at location (nbrSeg+1, nbrSeg+1).

\subsection{Time Stepping}

Using the above operator we will step in time using RK4.

In principle it produce something similar to this...


\begin{equation}
	\begin{bmatrix}
		Vs \\
		V^{k+1} \\
		I^{k+1} \\
	\end{bmatrix}
	=
	\begin{bmatrix}
		V_s \\
		V^k \\
		I^k \\
	\end{bmatrix}
	+
	\Delta t
	\begin{bmatrix}
		0 \\
		\frac{\partial{V}}{\partial{t}} \\
		\frac{\partial{I}}{\partial{t}} 
	\end{bmatrix}	
\end{equation}


\section{Python Implementation}
The program is named stltfdrk4.py for Single Transmission Line Transient Finite Difference runge kutta 4.\\

The signal injected is  gaussian pulse centered at 100ns with a Tau of 20ns multiplied by a triangular window of 200ns wide centered at 100ns.\\

\begin{figure}[H]
	\centering
	\includegraphics[width=1\textwidth]{gaussian_pulse.png}
	\caption{Gaussian Pulse}
\end{figure}


\begin{figure}[H]
	\centering
	\includegraphics[width=1\textwidth]{gaussianpulse.png} % Adjust width as needed
	\caption{Test pulse propagating, Rs=50, Rl=1}
\end{figure}

\begin{figure}[H]
	\centering
	\includegraphics[width=1\textwidth]{1000m.png} % Adjust width as needed
	\caption{13 MHz sinewave, coax=1000m, Rs=50, Rl=50.}
\end{figure}

\begin{figure}[H]
	\centering
	\includegraphics[width=1\textwidth]{step-1.png} % Adjust width as needed
	\caption{Step, cable 100m, Rs=50, Rl=open}
\end{figure}

\begin{figure}[H]
	\centering
	\includegraphics[width=1\textwidth]{step-2.png} % Adjust width as needed
	\caption{Step, cable 100m, Rs=50, Rl=50}
\end{figure}


\section{MFEM Implementation}

The program is named stltfdrk4.cpp for Single Transmission Line Transient Finite Difference runge kutta 4.\\

\begin{figure}[H]
	\centering
	\includegraphics[width=1\textwidth]{pulse_mfem.png} % Adjust width as needed
	\caption{Test pulse propagating, Rs=50, Rl=1, MFEM}
\end{figure}

\begin{figure}[H]
	\centering
	\includegraphics[width=1\textwidth]{1000m_mfem.png} % Adjust width as needed
	\caption{13 MHz sinewave, coax=1000m, Rs=50, Rl=50, MFEM}
\end{figure}

\begin{figure}[H]
	\centering
	\includegraphics[width=1\textwidth]{step-1_mfem.png} % Adjust width as needed
	\caption{Step, cable 100m, Rs=50, Rl=open}
\end{figure}

\begin{figure}[H]
	\centering
	\includegraphics[width=1\textwidth]{step-2_mfem.png} % Adjust width as needed
	\caption{Step, cable 100m, Rs=50, Rl=50}
\end{figure}

\section{Conclusion}
The telegrapher's equations are resolved using finite difference method first in python using numpy and scipy and then C++ using MFEM library.

\end{document}
